\documentclass{article}
\usepackage[utf8]{inputenc}
\usepackage{multicol}
\usepackage{geometry}
\usepackage{xcolor}
\usepackage{pgffor}

% pandoc syntax highlighting

\fontfamily{sans-serif}
\selectfont

\geometry{letterpaper, top=60pt, left=80pt, right=120pt}
\setlength{\columnsep}{2cm}
\definecolor{light}{rgb}{0.56078431373, 0.52941176471 ,0.52941176471}

\begin{document}

  {\noindent \huge Crema de huevos de pescado {[}tumbacatre{]}}
  \vspace{1cm}

  \begin{multicols}{2}
  \noindent \color{light}
      2 libras (1 kg) de huevos de pescado, limpias\newline
      8 tazas de leche de vaca\newline
      5 cucharadas demantequilla\newline
      1 plátano verde, pelado y ras­pado con una concha de piangua\newline
      1/2 tazade refrito {[}ver p.~50{]}\newline
        1 taza de leche de coco, espesa\newline
      sal y pimienta a gusto\newline
      2 cu­charadas de cebolleta, picada fina\newline
      1/2 cucharadita de picha de toro o de tortuga macho,\newline
    \end{multicols}
  \vspace{1cm}

  {\noindent \LARGE Instruciones}\\
  \\
  \noindent \color{light} seca y raspada.Se lavan y se secan los huevos, se pican y se ponen con
el sofrito, sal y pimienta enla mantequilla a fuego moderado por 10
minutos. Aparte, se pone la leche a hervira fuego medio, se le agrega el
plátano raspado y se deja cocinar por 15 minutos, sebaja a fuego muy
lento y se incorporan los huevos y la leche de coco; se re­vuelve,
setapa y se deja conservar por 10 minutos. Se espolvorea con la
cebolleta y el raspadode picha y se sirve.

\end{document}
